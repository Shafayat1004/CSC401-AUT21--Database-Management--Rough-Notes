\documentclass{article}
\usepackage{amsmath, amsthm, amsfonts, bm, graphicx, xcolor, soul, mathtools, parskip}

\title{CSC401 Lecture 1 Summary}
\author{Mir Shafayat Ahmed}

\begin{document}
    \pagecolor[HTML]{FFFFCC}
    \maketitle
    \section*{Rich Picture}
        \subsection*{Purpose}
            Why do we develop a software solution. First we have to find the problem. Analyse it.\\
            We will learn some analysis techniques. We start with Rich Picture to get a generalized idea.
        
        \subsection*{Starting}
            We will get a better understanding of the system when we start to make it.
            \\
            What we need to include is discussed.
            \\
            The green highlighted area is the existing system.
            The other stuff are external.
            \(\rightarrow\) uses shown.
            We draw rich picture to see where we can automate already existing processes.
            \\
            Dialog boxes discuss the issues. The arrows describe the process. Logos are used to represent people and structures.
            \begin{equation*}
                Doctors \xrightarrow{treat} patients
            \end{equation*}
            An issue of Doctors are they can't remember stuff long time ago. So that would be expressed in the dialog box.        
            \\
            In another type of rich picture rather than have arrows, we list the activities performed by the structure or people below.
            \\
            We use a dashed arrow \(\dashrightarrow\) where direct participation is not required. Also if there is any room for vagueness we use dashed lines. 
            \\
            Bidirectional arrows \(\leftrightarrow\) are used where both actors participate. Both entities are equally important.
        \subsection*{Model 1: Process Flow}
            A student enters the library \(\rightarrow\) Searches for book \(\rightarrow\) Finds book \(\to\) register books \(\to\) RFID reader \(\to\) local pc searches database for availability \(\to\) etc
            \\
            But we wont use this, we won't represent process flow in rich picture. We will do that in BPMS diagrams.
        \subsection*{*Advice for group given*}
        \subsection*{What to do}
            Rich Picture can be less detailed. But making a detailed picture will help the analysis in later stages.
    \section*{System Element Analysis}
        A system is a combination of tasks done by entities to achieve some goal.\\
        We will be working with Information System.
        \subsection*{Information System (IS)}
            Arrangement of people, data, processes and IT. Where IT is a term where we use computer tech (hardware and software) with telecommunication technology (data, image and voice networks etc.).
        \subsection*{The Six Elements}
            Human, Non-Computing hardware, computing hardware, software, database, communication and network.
            \subsubsection*{Human: Stakeholders}
                Owners, Users, Dev Team, External Service Providers.\\
                Stakeholders consist of:
                \begin{itemize}
                    \item Primary (Primary Customer)
                    \item Secondary (People responsible to develop the system (Dev Team))
                    \item Key (System owners, key decision makers, goal setters, authority)
                \end{itemize}
            \subsubsection*{How to Identify Stakeholders}
                Contact people after brainstorming, research work.
                For a 6 month project, we will do most rigorous analysis within 15 days.But analysis doesn't stop. it is an ongoing process.The requirements might change.
                \paragraph
                All analysis for the course project should be done within 5 days.
                \\
                2 days for existing system analysis. Show to client. Fix.
                \\
                Next 3 days for proposed system. Show to client. Fix.
            \subsubsection*{Non-Computing Hardware}
                Forms, Reports, Books, Printed Stuff etc.
                We need to know if people use them and how they use them.
            \subsubsection*{Database}
                *Log files*, Register Book, SQL Server
            \subsubsection*{Why use Six element analysis}
                Reduce non computing elements.
            \subsubsection*{Basic Data Entry}
                For each element, we list out who/what falls under the category. For each we show what they do or what things are used for. Step by step activities must be mentioned that work towards a goal.
    \section*{Advice}
        Always focus on your goal. Do not mention things irrelevant to you.\\
        Take max 5 days after project details are given for rigorous analysis.\\
        Identify around 15 Processes.\\
        

            
                

        
    

\end{document}