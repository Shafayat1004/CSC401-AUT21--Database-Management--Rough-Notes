\documentclass{article}
\usepackage{amsmath, amsthm, amsfonts, bm, graphicx, xcolor, soul, mathtools, parskip}

\title{CSC401 Lecture 2 Session 2 Summary}
\author{Mir Shafayat Ahmed}

\begin{document}
    \pagecolor[HTML]{FFFFCC}
    \maketitle

    \section{BPMN}
        \subsection{Swimlane}
            What does the entity do? What are the activities performed.\\
            \emph{Pools} represent a process from the entities perspective. Start to end.\\
            For example, in a Doctor's Office Pool there are two separate roles. Receptionist and Doctors Lanes are created.
            
            \paragraph{}
            \emph{Sequence Flows} can cross lanes.\\
            \emph{Subprocesses} are rounded rectangles withing rounded rectangles. Within Subprocesses there doesn't need to be a start-end event. But a normal process flow must have a start-end event.
            
            \subsubsection{Pool}
                \emph{Dashed Lines} crossing boundaries are message flows. Sequence Flows cannot cross boundaries. The process \emph{is} the sequence flow.\\
                In a message, if there is a specific data being sent, we add the \emph{Data-Object} symbol. We connect the \emph{Data-Object} with the \emph{Association (Dotted Line)} connecting to the \emph{Message Flow Line}
            
            \subsubsection{Lanes}
                To transfer a data object we use a directional \emph{Association} arrow with the Data-Object in the middle.
            
            \subsubsection{Subprocesses}
                Messages can go out of subprocesses. Sequences cannot.
    \section{Improvement Process}
        The idea stems from business Process Engineering. Our target is always to increase quality, decrease time and cost (QTC).
        
        \subsection{Key Performance Indicators (KPI)}
            \subsubsection{}



\end{document}