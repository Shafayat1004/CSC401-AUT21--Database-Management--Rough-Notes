\documentclass{article}
\usepackage{amsmath, amsthm, amsfonts, bm, graphicx, xcolor, soul, mathtools, parskip}

\title{CSC401 Lecture 2 Session 1 Summary}
\author{Mir Shafayat Ahmed}

\begin{document}
    \pagecolor[HTML]{FFFFCC}
    \maketitle
    \section*{Business Process Modeling}
        We make Rich Picture firstly then we move to Six Element Analysis. SEA is more detailed than Rich Picture. After that we go into further analysis with Business Process Modeling.
        For each Element, there should be separate models.
        \subsection*{BPMN 2.0}
            We follow the BPMN standard and therefore the shapes of Flow Objects:
            \begin{itemize}
                \item Events (Circle)
                \item Activities (Rounded Rectangle)
                \item Gateways (Diamond)
            \end{itemize}
            Connectors: 
            \begin{itemize}
                \item Sequence Flow (solid line with filled arrowhead)
                \item Message Flow (hollow dot, dashed line, hollow arrow)
                \item Association (dotted line, open arrowhead)
            \end{itemize}
            Artifacts:
            \begin{itemize}
                \item Data Objects
                \item Text area
                \item Groups
            \end{itemize}
            Swimlanes.
            % figure here

            \paragraph{}
            Activities are Rounded Rectangles.\\
            Atomic \emph{Tasks} do not need to be broken down.
            Activities that can be broken down to smaller tasks are called \emph{Sub-Processes}.\\
            Activities that are repeated, are called \emph{Looped Task}.
            \emph{each type of tasks have specific shapes keep in mind}\\
            For tasks that will be performed by users, an icon of the user will be given o the top-left corner.\\
            Specialized tasks markers are in top-left. \emph{Not Mandatory}\\
            \subsubsection*{Events}
                \begin{itemize}
                    \item Start (simple circle)
                    \item Intermediate (double circle)
                    \item End (bold circle)
                \end{itemize}
                \textbf{\emph{Event is something that \emph{Happens}.
                Activities are done by some entity}}
                %  insert image of different Event Triggers
                \begin{itemize}
                    \item None (Undefined/Unknown)
                    \item Message Based
                    \item Timer Based
                    \item Rule Based
                    \item Multiple reasons based Triggers (Either OR)
                    \item Link 
                \end{itemize}
                Intermediate Events may have some more.
                \begin{itemize}
                    \item Error
                    \item Compensation
                \end{itemize}
                Sometimes a sequence of tasks can be interrupted.
                We represent it by placing an intermediate event in the boundary of a task.
                \paragraph{}
                For End Events, we usually don't think of them as triggers. We think as if something is given at the end.\\
                Timer and Rule based products do not appear in End Events generally:
                \begin{itemize}
                    \item Terminate
                \end{itemize}
            
        \subsection*{Gateways}
            \begin{itemize}
                \item Exclusive (We can take one and only path) (X can be within the diamond)
                    \subitem Data-based
                    \subitem Event-based
                \item Inclusive (one or many paths can be taken)
                \item Complex (something else)
                \item Parallel (Multiple pathways are followed parallelly)
            \end{itemize}
            \subsubsection*{Exclusive, based on data}
                A hatch mark / means default path.
            \subsubsection*{Inclusive Gates}
                Since we can take multiple paths, we always use two inclusive gates that will eventually merge the paths.
            \subsubsection*{Parallel}
                They do not need a condition. All paths are taken simultaneously. Because of different paths, they will need to be converged so here too, we always need two of them.
            \subsubsection*{Sequence flow}
                connects flow objects.
            \subsubsection*{Message Flow}
                Passing of messages from one pool to another. Label must be present.
        
        \emph{Swimlanes and part 2 of this topic will be in Lecture 3}
            
                

                


            

    

\end{document}